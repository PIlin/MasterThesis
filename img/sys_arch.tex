%!TEX root = ../master.tex


\begin{tikzpicture}[auto, node distance=6em,]

	\tikzstyle{block}=[draw, text width=5.7em, text centered, minimum height=3em, fill=white]

	\begin{pgfonlayer}{foreground}
		\node[block] (hc) {Серверний контролер};
		\node[block, below of=hc] (ha) {Серверний додаток};
		\node[block, right of=ha, node distance=10em] (ra) {Вiддалений додаток};
		\node[block, right of=hc, node distance=10em] (ln) {Клієнт};

		\draw[thick,<->] (hc.south) -- (ha.north);
		\draw[thick,<->] (ha.east) -- (ra.west);
		\draw[thick,<->] (hc) -- (ln);

		\path (hc.west) -- (ln.east)
			 node[pos=0.5](betw-hc-ln) {} ;

		\node[block, above of=betw-hc-ln, text width=10em, node distance=4em] (sen) {Сенсори};

		\draw[thick,->] (sen) -- (hc);
		\draw[thick,->] (ln) -- (sen);
	\end{pgfonlayer}

	\begin{pgfonlayer}{background}
		\path (hc.west |- sen.north)+(-0.5,0.9) node (a) {};
		\path (ha.east |- ha.south)+(0.5,-0.3) node (b) {};
		\path[draw] (a) rectangle (b);
	
		\path (ln.west |- sen.north)+(-0.5,0.9) node (c) {};
		\path (ln.east |- ln.south)+(0.5,-0.3) node (d) {};
		\path[draw] (c) rectangle (d);
	\end{pgfonlayer}

	\begin{pgfonlayer}{foreground}
		\draw[] (a) node[anchor=north west] (host-label) {Сервер};
		\path[draw] (host-label.south west) -- (host-label.south east) -- (host-label.north east);
		
		\draw[] (c) node[anchor=north west] (node-label) {Клієнт};
		\path[draw] (node-label.south west) -- (node-label.south east) -- (node-label.north east);
	\end{pgfonlayer}

\end{tikzpicture}
